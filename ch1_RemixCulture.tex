\chapter{Remix Culture}
\label{ch:ch1_RemixCulture}

% --- Chapter 1 start ---

Remix Culture is the overarching theme of this dissertation. This phenomenon encompasses a large number of domains and aspects of everyday life involving creativity, thus shaping the way these works are being created, shared, used, and reused. Indeed, the fruition and the reuse are the most characterising facets of the Remix concept.

As mentioned in the introduction, Remix Culture is a transformative practice. By taking one or more pieces of existing items – both physical and digital – new works can be created. For instance, the replacement of an audio source within a certain video or more complex arrangements like entire movies made by using other audio visual elements are both valid examples of transformative practices. These kinds of remixes can radically change the meaning and the cultural perception of the original works.

It appears that the Remix Culture can be seen from two high-level perspectives, the first tied to the digital evolution. Indeed, among the inspirations of the Remix Culture is the Free and Open Source-Software (“FOSS”), an initiative which will be introduced in paragraph \ref{sec:IPR} \emph{Open Source} as a precursor to the Open Source movement. In short, the goals and ideas of these movements are mainly oriented towards encouraging a “free” distribution of software. “Free” is a vague term that will have to be discussed later in more details.

A second perspective is not exclusively tied to the digital evolution, rather it can be seen as a sociocultural concept that affects several aspects of human behaviour. From a broader perspective, the Remix Culture can be thought of as a way the society works and evolves through history. The idea can be explained by using the popular metaphor of “standing on the shoulders of giants” in the sense used and intended by Isaac Newton\footfullcite{bbcNewton}. Therefore, the ability to discover and enhance knowledge – hence progress as humans – can be reached thanks to the contributions of past and present human generations with the knowledge that had been passed to us.

This concept is further sustained by Lev Manovich in his article “Remixing and Remixability”\footfullcite{ManovichRemixability}. The author argues that:

\begin{displayquote}
    “most human cultures developed by borrowing and reworking forms and styles from other cultures; the resulting “remixes” were to be incorporated into other cultures. Ancient Rome remixed Ancient Greece; Renaissance remixed antiquity; nineteenth century European architecture remixed many historical periods including the Renaissance…”.
\end{displayquote}

This excerpt suggests a definition of remixability as a common trait of human behaviour as deducted from a historical perspective. This can be generalised to humans as communities\footfullcite{wikiSenseOfCommunity}, as well as individuals. The latter view becomes the core point of author’s subsequent analysis when describing the modern times. The core point of his argumentation is the major novelty introduced by the Internet, that is the unprecedented level of participation of individuals with the relative ease of accessibility. Indeed, related to this aspect, he states that “culture has always been about remixability, but now this remixability is available to all participants of Internet culture”\footfullcite{ManovichRemixability}. Nowadays, the pervasiveness of Internet connectivity has led to a professionalization of individuals. Thanks to information available on the web combined with software tools that allow to produce new content, the individuals can become active members of the Remix Culture. Finally, acting as creators, members of the public are able to share their work almost instantly, everywhere and at little or no cost. 

As a matter of fact, this reflects the tendency of new social interactions enabled by the Internet. Many of the most popular online platforms, like Youtube, TikTok, Instagram, etc. make extensive use of the Remix Culture. In the next section \ref{sec:RmxExamples} \emph{Examples of Remixes}, some practical examples of remixes from various disciplines and time frames, ranging from texts, music, films to videogames will be presented to better illustrate the whole phenomenon.

Before expanding further in framing Remix Culture, a first glance on common issues can be taken by reflecting upon this rather radical change of paradigm. Hence, the change from an analogic and physical to a digital environment. It could be argued that the national and international laws have not responded adequately to this change of paradigm. According to Lawrence Lessig, who may be considered among the most prominent figures in favour of this argument, states that: 

\begin{displayquote}

“For the first time, the [copyright] law regulates ordinary citizens generally. For the first time, it reaches beyond the professional to control the amateur— to subject the amateur to a control by the law that the law historically reserved to professionals.”\footfullcite{LessigRemixMA}

\end{displayquote}

This suggests an underlying problem from a legislative perspective, thereby the copyright law. Further evidence will be provided in paragraph \ref{sec:IPR} Intellectual Property Laws and Reusability. The overall point is that by analysing the past and current evolution of works of creativity made by “amateurs” i.e., general public, it seems that the law has taken some stringent measures to limit the Remix phenomenon. This is also true for professionals and large businesses. Additionally, some uncoordinated decisions have been made to respond to certain issues with the practical applications of copyright laws. This overall might be considered as an inhibitory factor acting against the Remix Culture ideas and practical applications.

Another interesting point of view related to the concept of Remix Culture is given by the aforementioned Lawrence Lessig, the founder of Creative Commons. In his book “Remix Making Art and Commerce Thrive” he defines Remix Culture as a Read/Write (“RW”) culture as opposed to a Read/Only (“RO”) culture. The author uses an analogy of how popular computer systems work (for example Linux based systems, Windows etc). Basically, users with read and write permission on a file or directory are authorized to read it and make changes. On the contrary, the read only permission allows only for files and directories to be read disabling the ability of making any modifications.

Returning to the cultural context, the latter is the situation where the users or the general public passively consume the content made by someone else. This is frequently the case of professionally made content, backed by expensive tools and potentially complex organisational structures. For example, the TV broadcasts, production of high budget movies and their subsequent distribution on CDs, DVDs, etc. It could be argued that the Read/Only culture lasted until the times of Internet pervasiveness, as it is characterised by a top-down consumption of cultural objects. On the other hand, the Read/Write culture allows for a democratized participation on the creative process. Hence, a democratization of the act of creativity where people participate in the creation and the re-creation of culture.
These arguments represent a paradigm switch. In the Read/Only culture information flows one-way – unidirectionally –, while in the Read/Write culture the information is multi directional or, speaking in terms of networking: peer to peer like\footfullcite{wikiPeerToPeer}.

At this point, it should be clear that the technological progress allowed to minimize the gap between the professionally created content and the works of creativity that can be made without substantial investments or infrastructure. This can be accomplished thanks to software that makes the technical operation of “remixing” relatively easy.

This concept is also exemplified by Lev Manovich in relation to music remixes and music mashups. In his article “Remix and remixability” he states that: “Although precedents of remixing in music can be found earlier, it was the introduction of multi-track mixers that made remixing a standard practice. \footfullcite{ManovichRemixability} \footnote{This specific example is somewhat related to this dissertation’s practical application. Instead of music as exemplified by Lev Manovich any type of media can be put into a multi-track editor to create new remixes.}”

Hence, what really made the Read/Write culture come to life was the 21st century digital revolution. Indeed, there are some relevant differences between the way things could be created, re-used, and shared before the age of Internet. A notable example showing these differences is the invention of the mechanical movable-type printing press, that is an efficient machine for printing texts. The main difference between the printing press and the Internet is that the printed output copies were inferior to the originals in terms of quality. The affordability was also an issue, costs related to printing a copy prevented people to fully benefit of that invention. Another aspect of this problem is that people had little choice regarding the production of copies for themselves. The alternative choices – until the introduction of low-cost home prints – would imply an inferior quality to those coming from the professional printing businesses. Thus, the whole process relied on professionals.
On the other side, the Internet can be seen as the most efficient, low-cost copying and sharing mechanism accessible from everywhere. Furthermore, the sharing process allows for resources to be distributed globally between connected users, and this is a substantial difference from the previous models. Therefore, users do not often need to rely on professionals to make creative works.

These differences can be further explained from an economical perspective as the distinction between rival goods and non-rival goods\footfullcite{wikiRivalry}. In general, most physical goods are rival goods. For example, a sandwich is a rival good, because the act of eating it clearly diminishes its value. A sandwich loses its value while it is being eaten, so to speak. Sharing rival goods implies the that the benefits of use are decreased or eliminated, so sharing a piece of sandwich produces a tangible loss for the owner of the sandwich. Analogously, sharing a printed copy of a book is an example of rivalry because it prevents the original owner from its unlimited consumption.

On the other side the definition of non-rival goods implies that: “… one person’s enjoyment of a good does not diminish the ability of other people to enjoy the same good.” \footfullcite{KotchenPublicGoods}. Hence, ideas and words tend to be non-rival. Sharing them does not make them worse or less valuable. An exception of a physical good that is non-rival could be a public bench. The bench value is not significantly diminished when used by people (at least, hopefully, if it is being used correctly). From this perspective a large majority of digital goods are non-rival. A relevant example is an e-book, no one ever “bought” an e-book, people buy a license to read e-books on their devices of choice. Namely, the use of copies does not preclude its accessibility from others and is not directly connected to tangible losses. A few exceptions like domain names and so on exist in this category as well.

Expanding upon the non-rivalry definition, in his book “The Success of Open Source (2004), Steve Weber argued that some ideas, words, etc. are definable as “anti-rival goods” meaning that they are improved by being shared in a manner similar to the economics idea of the “Network effect”\footfullcite{wikiNetworkEffect}. Some examples could be the case where a social media gets more powerful as more and more people are using it, or an article becomes more valuable as its hyperlink is being shared across the web, etc. In short, the act of sharing increases the benefits for other participants. For this same reason it could be argued that the efforts to combat climate change are non-rival because the benefits of these actions are shared among all the word’s living species, including for example all the nations who refuse to do so. 

This final point of analysis suggests that the Remix Culture could be in fact non-rival. This argument could be rephrased as a direct question, “are the works of creativity produced as a result of remixing beneficial for the general public?”

Unfortunately, the proposed answer is murky, since from a general point of view it would be very hard to answer with a simple “no” or “yes”. It might be argued that it depends on case to case and on the relative contexts where this question is made. Probably, the answer might not always be positive from the perspective of original creators or de facto owners of works whom creations might be used in a way that contradicts their believes, such as propaganda, commercial profits, etc. Nevertheless, the key point worth considering is that as a matter of fact, everything can be remixed and distributed globally.

In this view, a consideration on the PH-Remix project case is pertinent. Among the project’s goals are dissemination, discovery, use and re-use which may enhance the value of created content. From a practical point of view, short clips are extracted from films uploaded to the platform – mainly documentaries at the time of writing – using Artificial Intelligence algorithms. Subsequently, in an idealistic scenario all the clips should become available for remixing i.e., being arranged and combined with other clips to form new audio-visual creations.

However, the tendency of the film industry could be considered sceptical about these ideas or simply not fully aware about the possibilities and new ways of innovating provided by digitisation. Considering the nature of the content initially put on the platform, it could be argued that cultural heritage works and especially documentaries should oblige a moral and ethical motivation of giving back to the communities and not be closed in organisational siloes protected by stringent laws. Indeed, PH-Remix for cultural heritage would be a case of non-rivalry because it should be desired to share the messages and stimulate a debate with as many people as possible with as little limits as possible.

Naturally, critiques about the idea of Remix Culture exist\footfullcite{rmxWoRomance}. They will be further discussed in the next paragraph in relation to copyright issues. It could be argued that a relevant part of human progress alongside with some of the most iconic inventions were made thanks to the presence of copyright laws. Nowadays, by taking into consideration that the cost of distributing content is as low or almost nonexistent for digital goods it may still be necessary to reward the creators of intellectual property.

This objectively seems to be true. Remix culture can be enabled by openness and open source licenses, but these are not a silver bullet for all projects and creations. A relevant example was once provided during the “Open Cultures” course at King’s College London by Dr John Lavagnino. He argued that J.K. Rowling could probably not have finished the Harry Potter books if she had decided to start publishing them without a license. A source of income is often needed and necessary. On the other hand, the more people who read Harry Potter the better. Generally, as a creator you probably would want to get as many people as possible to watch/be able to consult your creation. In chapter \ref{ch:ch2_ProjectManagement} \emph{Project management in interdisciplinary projects}, arguments advocating for a sustainable economy for content creation – especially for cultural heritage – will be discussed in more detail. 

\section{Examples of Remixes}
\label{sec:RmxExamples}


The practice of modifying both physical and digital items has a long-standing tradition. Remix applies to music, movies, cooking recipes, software, and many other disciplines. After introducing the Remix Culture mainly from a theoretical point of view, it is useful to make some practical examples to better understand the overall concept. For this reason, a selected list of examples from different domains – although mainly focused on the modern age time frame – is presented in this section.

An in-depth study of the creative processes and the way content is being re-used was done by Kirby Ferguson in his four-part video series titled “Everything is a Remix”\footfullcite{everythingIsARemix}. Indeed, he states that the acts of copying, transforming, and combining are the basic elements applicable at any level of creativity. Hence, the following assertion: “creativity is not magic, it happens by applying ordinary tools of thought to existing materials”\footfullcite{everythingIsARemix3}.

Music mashups are one of the most common examples of such transformations. Specifically, hip hop music was one of the first musical forms to incorporate samplings in the recordings\footfullcite{wikiMashup}. Thereby, the results are creative works which typically incorporate fragments of other songs. These fragments are generally re-arranged, thus transformed to produce new sounds and songs. Indeed, in most cases mashups are legal by considering them under the various boundaries of “fair use” law doctrine. These boundaries are usually fuzzy although the tendency of re-using pieces of other songs is very popular among many artists.

One example of a band which crossed boundaries of fair use is Led Zeppelin. They copied significant part of other songs without making fundamental changes. Two documented examples of songs subject to legal claims can be seen on the list below:

\begin{itemize}
\item Led Zeppelin song: “Bring it on home” was a copy of Willie Dixon – “Bring It On Home”\footfullcite{ledZeppelin1}
\item Led Zeppelin – “Stairway to Heaven” most likely copied from the band Spirit with their song “Taurus”\footfullcite{ledZeppelin2}
\end{itemize}

Another discipline which makes extensive use of Remix is the cinema industry. Many movies are inspired by the surrounding works of culture. Nowadays, a common technique consists in transforming the “old” into the “new”. That means taking or re-creating already existing materials from literature, actual events, etc. and producing movies for the current generations. Even existing movies are often a base for new cinema adaptations and frequent prequels and sequels.

One important example of remixing in the film industry are the Disney movies. The Walt Disney Company made extensive use of works from the public domain. Some examples from their repertoire of animated films history are listed below:

\begin{itemize}
\item The Little Mermaid is based on the “The Little Mermaid” fairy tale written by Hans Christian Andersen.
\item Alice in Wonderland is based on the novel “Alice's Adventures in Wonderland” and its sequel, “Through the Looking-Glass” both authored by Lewis Carroll.
\item Aladdin is based on the folk tale “Aladdin from the Arabian Nights” originated from the Middle Eastern culture and later interpreted by Antoine Galland.
\item Mulan is based on the traditional Chinese story of “Hua Mulan”, a legendary folk heroine. 
\end{itemize}

Once transformed into animated movies, Disney likewise any other film producer, were entitled to a period of exclusivity on their productions thanks to copyright. Then after a certain period their work is supposed to enter the public domain to be freely used and built upon. This case was regulated by the American legislation in the Copyright Act of 1976\footfullcite{wikiUSCopyrigh1976} which established the duration of copyright for the life of the author plus 50 years, or 75 years for a work of corporate authorship.

Interestingly, some companies and prominently Disney lobbied to have their terms of copyright extended\footfullcite{wikiUSCopyrighExtensionAct} with the Copyright Term Extension Act, also known as “Mickey Mouse Protection Act”. The company decided to prevent others from copying its works. Therefore, standing to the current copyright terms, Mickey Mouse, which firstly appeared in 1928, will enter the public domain starting from 1st January 2024\footfullcite{mickeyMouse}. It could be argued that by trying to foresee into the future, Disney will attempt to prevent it from happening by using other laws including trademark protection and using precedents in their favour from other court verdicts.

Returning to the practical cases, the film industry is a rich source of similar examples. For instance, an in-depth analysis of the Star Wars series results in finding multiple references and inspirations from historical events, fiction literature, etc. as well as some copied elements from other film productions. It could be argued that without the influence of past creations Start Wars would not have been created.

The digital evolution alongside with the birth of software for video editing allowed for a more accessible and complex remixing of new content outside of the professional world. Fan-made trailers, remixes, memes that spread virally are all consequences of the ease of use enabled by a mix of modern technology combined with user’s creativity.

Certainly, a long list of important examples could follow. For example, Machinima\footfullcite{machinima} productions, videos generated from video games graphic engines are a relevant contribution to re-use.
The core point is that the strength of these creations is also traceable to the communities of users that participate in the creative process and the consumption. Indeed, there are several examples of Remixes as community efforts.

Perhaps the most well-known and successful community effort is Wikipedia. Wikipedia’s main power is a huge community of active volunteers thanks to which it was able to succeed.

Other projects like Free Beer\footfullcite{freeBeer} and OpenCola\footfullcite{openCola} are examples of remixing in the physical world. They respectively consist in creating and improving recipes for beer and (coca) cola and encouraging their production by adopting a permissive license to the recipes.
In reality, the latter example is connected to the Open Source community as an attempt to replicate the concepts of free sharing and contributing into the physical world which was already a common practice in the software world. 

Naturally, Remix could be viewed as a form of art. Like the latter it is also subject to personal interpretation because, among its goals, is the creation of spaces for debate. Obviously, they often might be a source of tensions and critiques, as the example below.


\begin{figure}[H]
\centering
\includegraphics[width=0.6\textwidth]{images/arteAcquaSalvini.jpg}
\caption{"Arte dell'acqua" by students of Russoli High School (Pisa)}
\label{fig:arteAcqua}
\end{figure}

Figure \ref{fig:arteAcqua} "Arte dell'acqua" is a collage of approximately 400 photos of migrants which depict some tragedies happening at sea while trying to reach the coast. Together the pictures form the face of an Italian politician, Matteo Salvini. He is known for his ideological line against illegal immigration as he advocates for stronger actions and policies to enforce its prevention. Indeed, these forms of remixes can inevitably create controversies. Nevertheless, art encourages reflection and debate and above all it is an expression of free speech.

Lastly, copying is not necessarily equivalent to plagiarism. For instance, this whole dissertation could be seen as a remix. Originated by combining and arranging a set of factual sources with quotations alongside with the author’s academical background, personal experiences, and cultural bias to produce something new. 
While this last example of remix is perfectly legal – if it complies with some rules about academical quotations, etc. – and thus cannot be considered as a plagiarism, the same cannot be said about other contexts and media types.


\section{Intellectual Property Laws and Reusability}
\label{sec:IPR}

Understanding the concepts related to intellectual property (“IP”) and how they are regulated, is fundamental in order to acquire a complete picture of the Remix Culture.

Naturally, the activity of Remix is deeply interconnected with a multitude of laws. It is fair to debunk upfront that an extensive analysis of the topics presented in this section is beyond the scope of this work.  These are complex topics where legal terms involve the use of legal technicalities which differ from country to country. However, understanding the fundamentals of concepts like copyright and licensing is fundamental to explain and to illustrate some frequent problems occurring when adopting the Remix Culture.

As introduced in the previous paragraph, the re-use of content can be substantially limited and potentially discouraged because of the legal bounds applicable to a multitude of physical and digital goods. These legal bounds are typically expressed in form of author’s rights – or copyright, as these terms are used interchangeably in this dissertation – and can be contained in licences applicable to different kinds of works. In short, by creating new works authors are automatically entitled to exclusive rights on them and, as long as a permissive license is not applied, these works cannot be freely reused by others.

In reality, this might often work as a chilling effect, that is, “a discouraging or deterring effect, especially one resulting from a restrictive law or regulation” \footfullcite{collinsChillingEffect}. Furthermore, this effect can also be attributed to the quantity and the complexity of regulations affecting the specific items that might be of interest for re-use. A relevant argument in favour of this assertion comes from a phenomenon called proliferation of licenses\footfullcite{wikiLicenseProliferation}. This is originated from the creation of a large number of similar agreements that might require some domain specific knowledge for their interpretation. Additionally, licenses – whose main goal is to clearly explain the rights and limitations applied to a specific object – are often incompatible between each other, even if they are similar in their permissiveness or strictness.

The subsequent analysis will be focused on illustrating an overview and developing an understanding of the legislative implications of some legal terms that are relevant for the Remix Culture.

Firstly, “Intellectual Property Rights” (“IPR”) is an umbrella term for a category of rights which regulate intangible creations of the human intellect\footfullcite{forbesIPR}.

\begin{displayquote}
“The main purpose of intellectual property law is to encourage the creation of a wide variety of intellectual goods. To achieve this, the law gives people and businesses property rights to the information and intellectual goods they create, usually for a limited period of time.”\footfullcite{IPRightUses}
\end{displayquote}

As seen in the above explanation, the reason for restricting the IPR duration to a limited period of time is mainly motivated by the goal of encouraging further creations. Hence, during the time of “exclusiveness”, authors can earn profit which in turn acts as an economic incentive for creation in the first place.

Secondly, speaking purely in terms of IPR might be considered as an overgeneralization. Richard M. Stallman argues that including different sets of laws under the IPR term is misleading\footfullcite{IPRGNU}. The former frequently happens with within the language of news media. Nevertheless, the subject’s literature about IPR commonly includes a number of typologies of intellectual property belonging to different branches of law. Namely, rights, patents, copyright, industrial design rights, trademarks, registered designs, trade secrets and so on, also depending on the jurisdiction.

As introduced in the previous paragraph, Remix Culture could potentially be influenced by all of these laws. By taking into consideration the most common use cases a subset of IPR typologies can be taken for a deeper analysis. Specifically:

\begin{itemize}
\item Patents
\item Trademarks
\item Copyrights 
\end{itemize}

Patents are used for protecting inventions. Namely, creativity works that are new, not trivial (for experts in the subject/domain) and are useful, thus capable of being used in some kind of industry. Patents give their owners the right to prevent others from making, using, selling an invention without an explicit permission. For example, scientific or mathematical discoveries cannot be patented, as well as a literary, dramatic, musical, or artistic work and many others. On the other side new plant varieties, medicines, machines, innovative solutions to technological problems are generally valid targets for patents.
An ongoing debate regards the creation of software. The European legislation regulated by the European Patent Convention explicitly excludes the possibility of patenting “computer programs” as stated in Article 52 “Patentable inventions"\footfullcite{EPC}. Still, it is subject to interpretation by courts. Vice versa the American legislation tend to allow software patents.

Certainly, from the point of view of the Remix Culture, software patents could have long-term negative effects. Namely, as stated by the Free Software Foundation Europe, “they specifically inhibit the development of useful software by blocking compatibility and interoperability“\footfullcite{fsfe}.

Secondly, trademarks are some sort of signs that distinguish a company or a service from another. They are also frequently used for marketing purposes like branding, product recognition, etc. Logotypes are a classical example of a trademark. For instance, the Nike “swoosh”, the Apple “bitten apple” and McDonalds “golden arches” are among the popular examples of trademarks.

Finally, copyrights regulate that works cannot be copied without the explicit owner’s permission. This applies to any medium. For example, nobody can make a movie based on a book without obtaining the permission of the owner of the book copyright. Copyright can protect literary works, including novels, instruction manuals, computer programs, song lyrics, newspaper articles, and some types of databases. Perhaps the most important characteristic of copyright is that it does not have to be applied for. Everyone gets it even if not aware of it.

These three are just some selected examples that creators should be aware of when doing remixes. Interestingly, it seems that the just introduced concepts are similar between the physical word and the digital world. Therefore, it could be argued that they are a straightforward translation of legislative terms originated from the history rather than new concepts adapted for the 21st century. This can be explained by considering that the technological progress happened very fast and especially the Internet was an abrupt revolution. In his article “The fight to keep ideas open to all” James Boyle, professor of Law at Duke University, analyses the current restrictions of IPR in a digital environment. He also states that:

\begin{quote}
“The internet has dramatically lowered the cost of copying, including illicit copying. When the web was first weaved in the 1990s, intellectual-property owners found their property had, involuntarily, been turned into a common. Strong new copyright rules and draconian enforcement seemed to be necessary to tame the rebellious digital commoners and reclaim the level of control that had existed in an analogue world.”\footfullcite{economyst}
\end{quote}

As seen in the above quotation, there are aspects that may justify some actions and fears of the copyright holders. In the recent years, policymakers, faced by arguments and lobbying, decided to adopt, and extend a copyright model which currently may seem as more and more unfit for the technological progress.

Nowadays, this led to a growing quantity of issues. In the same article James Boyle affirms that our smartphones are covered by between 5,000 to 15,000 patents and up to 250,000 if considering all the related patents. The general idea of scholars and professionals advocating for a more liberal approach to digital right is that intellectual property in the actual form blocks innovation and can be harmful for the society.

\subsection{Derivative Works and Fair Use}

...

\section{Open Source}
\label{sec:OS}

Open Source does not mean “free” as this can a common misconception due to its cultural meaning.

\section{Creative Commons}
\label{sec:CC}

“A global network lowers the costs of “commons collaboration” close to zero. Some people will participate to showcase their credentials, or to build a user-base for consulting services, or just because sharing is fun and costs nothing. Examples include Creative Commons (a less restrictive copyright system, which I had a hand in launching); Wikipedia; Linux, an open-source operating system; and the massive amount of useful digital content created by volunteers” 