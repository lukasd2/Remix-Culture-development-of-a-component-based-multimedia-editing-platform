\chapter{Remix Culture}
\label{ch:ch1_RemixCulture}

% --- Chapter 1 start ---

Remix Culture is the overarching theme of this dissertation. This phenomenon encompasses a large number of domains and aspects of everyday life involving creativity, thus shaping the way these works are being created, shared, used, and re-used. Indeed, the fruition and the usage are the most characterising facets of the Remix concept\footfullcite{ManovichRemixability} \footfullcite{wikiSenseOfCommunity}.

\section{Examples of Remixes}

The practice of modifying both physical and digital items has a long-standing tradition. Remix applies to music, movies, cooking recipes, software, and many other disciplines. After introducing the Remix Culture mainly from a theoretical point of view, it is useful to make some practical examples to better understand the overall concept. For this reason, a selected list of examples from different domains – although mainly focused on the modern age time frame – is presented in this section.

\section{Rights, Licences and Reusability}

Understanding the concepts related to intellectual property (“IP”) and how they are regulated, is fundamental in order to acquire a complete picture of the Remix Culture.

\section{Open Source}

Open Source does not mean “free” as this can a common misconception due to its cultural meaning.

\section{Creative Commons}

“A global network lowers the costs of “commons collaboration” close to zero. Some people will participate to showcase their credentials, or to build a user-base for consulting services, or just because sharing is fun and costs nothing. Examples include Creative Commons (a less restrictive copyright system, which I had a hand in launching); Wikipedia; Linux, an open-source operating system; and the massive amount of useful digital content created by volunteers” 