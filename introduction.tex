\chapter*{Introduction}
\label{ch:introduction}

% --- Introduction start ---

The Remix Culture is a transformative practice that affects a whole spectrum of creative works. In the digital world, thanks to the Internet pervasiveness, everyone can become a content creator by using and arranging existing contents from various sources. This practice – called “remixing” – can radically change the meaning of the original works during the transformative passage from the “old” into the “new”. Nowadays, this is substantially limited by various laws related to the Intellectual Property Rights. On the other hand, whether it is legal or not, this is what commonly happens in the digital environment. People make various forms of remixes, for example, parodies, memes, mash-ups, and other derivative works. The production of these is enabled by software and stimulated by a number of social platforms which offer these new forms of interactivity. On this regard this dissertation also shows the development of a web application that aims to further explore the possibilities of remixing digital contents. More generally, it seems that remixing can also be extrapolated from a number of other disciplines and physical works, thus leading to a thesis which correlates this sort of behaviour to the human nature itself.

This dissertation aims at making an in-depth analysis of the Remix Culture phenomenon. The benefits and the potential new applications together with some factors limiting its expansion will be discussed in detail later on. Moreover, a review of the state of the art and some proposals about alternative solutions to copyright issues such as Open Source and Creative Commons are some of the topics that will be covered in the first chapter.

Subsequently, an overview from the management perspective with considerations about the sustainability of the Open Source organisations and some best practices for project development will be demonstrated in order to connect the theory to the practice.

Finally, a practical example of a web application that enables remixing was made. The project consisted in creating a multimedia editor for audio-visual content. Thanks to this solution users are able to explore and choose multiple media types, namely, videos, images, and audio items from a library of elements. These media can then be viewed, arranged between one another, and inserted inside a multi-tracked editor. Analogously to some popular solutions for video-editing, this allows for real-time preview of the combined elements. The peculiarity or the main software value proposition is that this application works directly inside all the modern browsers.
In particular, the development process will be documented alongside with the explanation of the technical choices and the current software tendencies. Interestingly, this last chapter also advocates for remixing due to the adopted architecture based on reusable components with encapsulated functionalities. From the software perspective the component-based approach allows for interoperability thanks to the adoption of a standardised Web Components technology. In turn, this opens the possibility of combining single components just as LEGO blocks or puzzles. Indeed, this sort of new combinations can be made to produce new applications.

The theme of this work is also connected to the “PH-Remix” project. The acronym stands for “Public History Remix” and it involves the creation of a platform with the subsequent ingestion of resources belonging to the “Mediateca-Toscana” regional foundation. This project’s technical solution aims at exploring new ways of content dissemination, discovery, use and re-use which may enhance the value and the public utility of the uploaded resources. Furthermore, an important element of innovation consists in the automatic extraction of short clips from the videos by using Artificial Intelligence algorithms. For instance, at the time of writing, the uploaded content consists mainly of documentaries. Once uploaded to the platform, the algorithms are able to perform the object recognition task. This task returns a set of recognised objects, for example, people, furniture, ships, animals and many more together with the video frames containing the recognised objects. Hence, these video frames or video clips become short thematic extracts of the original media. Ultimately, they can be offered to the users for exploration and arranged inside the multimedia editing component.

To summarise, the PH-Remix main objectives can be framed into the cultural heritage perspective of digitization of audio-visual resources. It is an attempt to enhance content fruition by encouraging innovative ways of user participation and content discovery. PH-Remix is an ongoing project which involves, at the time of writing, the activity of three researchers from three different areas. Namely: archive’s analysis and cataloguing, data extraction and artificial intelligence, user experience and user interfaces.
Although the development of the web-based multimedia editor originated from the requirements of the PH-Remix project, subsequently it was adapted to fit a general-purpose scenario with a solution that can be modified and expanded upon according to the use case. Nevertheless, the project will be considered as a study case across all the chapters of this dissertation.

To review, the dissertation is logically divided into three chapters.
The main theoretical concepts with examples and copyright issues are explained in chapter \ref{ch:ch1_RemixCulture} \emph{Remix Culture}.
Considerations about project management and Open Source sustainability are discussed in chapter \ref{ch:ch2_ProjectManagement} \emph{Project Management and interdisciplinary works}.
The application development process with application examples is inserted in chapter \ref{ch:ch3_ProjectDevelopment} \emph{Project development web application for multimedia editing}.
