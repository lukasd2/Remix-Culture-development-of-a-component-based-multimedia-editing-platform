\chapter*{Conclusions}
\label{ch:conclusions}

This dissertation shows the interdisciplinary nature of the Remix Culture phenomenon, highlighting its importance in the modern society. The definitions taken from the topic’s literature and examples from the real world were aimed at showing a holistic view of the argument. The purpose was to present an original interpretation of the narrative starting from the theory through organisational and business topics to end with the development of a practical application. The latter enables users to directly experience the Remix Culture with as few technological limits as possible.

The application developed for this dissertation is a prototype, but the development process already provided some valuable skills and ideas that led to considerations about some best practices and arguments in favour of a standardised technology for creating components in the front-end ecosystem.
As a matter of fact, one of the discussed ideas regarded the possibility of remixing software thanks to a component-based approach. Although this is already possible, a substantial margin for improvement is also possible and desirable. Namely, the Web Components that are shared and published by the community do not often exactly work in a plug and play manner. At the time of writing, building an application that uses existing external components – built using different libraries and for different purposes – is not as easy as it would seem in theory. Notably, some weak points are to be reconducted to the lack of a comprehensive documentation guiding through all the installation steps, but some difficulties with interoperability are also common. In general, it seems that many existing components are not very developer-friendly.
Nevertheless, this approach seems promising and can be considered as a valuable asset for the current and future software developers with the hope that remixing components into new applications will proliferate. These considerations also lead towards some ideas for future works and studies that may further explore the criteria and effective ways for sharing and re-using software code.

The other critical point that emerged during the investigations was connected to the Intellectual Property Rights and their relation to the activity of re-use. Standing to current legislation many scholars and researchers advocate in favour of adapting the laws to the digital era. At the same time, the analysis of the Open Source and Creative Commons was presented to demonstrate viable alternatives to the current legislative issues and uncertainties. However, it was also demonstrated that these open approaches should not be considered as an ideal solution for every project and creative work. They are rather to be considered as an additional asset that can be used for creating sustainable hybrid economies where Open Source solutions are mixed with the proprietary ones guided by copyright rules.

Finally, one important goal of this works consisted in spreading awareness about the possibilities that the open world and permissive licensing offer. This is particularly important for some disciplines involved with the creation of art and culture. At the time of writing the GLAM (Galleries, Libraries, Archies and Museums) institutions are frequently involved in digitisation strategies that also include finding new ways to encourage content dissemination and user participation. The recent events, specifically the pandemic crisis, further showed that these innovations strategies are crucial for the culture sector.
From this perspective the “PH-Remix” project seems like an ideal candidate to explore the advantages of the Remix Culture when adopted by the cultural heritage organisations. While it is too early to decide whether the impact of the project will encourage and transform the traditional ways of content fruition, the result of the work seems promising and worth investigating once the project becomes fully operative.

To conclude, this dissertation showed some points of interest and possible topics for future research. The narrative was guided by an interdisciplinary approach. Hopefully, the latter may provide answers to some pressing questions that have not yet been solved by using more generic discipline-oriented approaches. It is also to be hoped that it will ultimately help in solving some of the current issues, thereby have a positive impact in transforming the way the society sees and uses the Remix Culture.